% Appendix B

\chapter{Importando código en LaTeX} % Main appendix title

\label{AppendixB} % For referencing this appendix elsewhere, use \ref{AppendixB}

Un ejemplo del syntax highlight predeterminado para Python:
\begin{lstlisting}[language=Python]
import numpy as np
 
def incmatrix(genl1,genl2):
    m = len(genl1)
    n = len(genl2)
    M = None #to become the incidence matrix
    VT = np.zeros((n*m,1), int)  #dummy variable
 
    #compute the bitwise xor matrix
    M1 = bitxormatrix(genl1)
    M2 = np.triu(bitxormatrix(genl2),1) 
 
    for i in range(m-1):
        for j in range(i+1, m):
            [r,c] = np.where(M2 == M1[i,j])
            for k in range(len(r)):
                VT[(i)*n + r[k]] = 1;
                VT[(i)*n + c[k]] = 1;
                VT[(j)*n + r[k]] = 1;
                VT[(j)*n + c[k]] = 1;
 
                if M is None:
                    M = np.copy(VT)
                else:
                    M = np.concatenate((M, VT), 1)
 
                VT = np.zeros((n*m,1), int)
 
    return M
\end{lstlisting}

Un ejemplo del syntax highlight personalizado para C++ (ver main.tex):

\begin{lstlisting}[style=dark_atom]
//----Importing libraries----//
#include <Wire.h>
#include <stdio.h> // I/O management i.e. sscanf()
#include <stdlib.h>

const int x_axis = 1;
const int y_axis = 2;
const int z_axis = 4;

void setup(void){
    Serial.begin(115200);
    Serial.println("Inicializando...");
}

void loop(void){
	x_val = calculate_average(x_axis);
    Serial.print("x_val = ");
    Serial.println(x_val);
    y_val = calculate_average(y_axis);
    Serial.print("y_val = ");
    Serial.println(y_val);
    z_val = calculate_average(z_axis);
    Serial.print("z_val = ");
    Serial.println(z_val);
}

int calculate_average(int axis){
    int value;
    int num = 10;
    for(int i = 0; i < num; i++){
        value += analogRead(axis);
    }
    int average = value/num;

    return average;
}
\end{lstlisting}